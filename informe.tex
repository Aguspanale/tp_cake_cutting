\documentclass[11pt,a4paper]{article}
\usepackage[spanish]{babel}
\usepackage[utf8]{inputenc}
\usepackage[T1]{fontenc}
\usepackage{amsmath,amssymb}
\usepackage{graphicx}
\usepackage{booktabs}
\usepackage{geometry}
\geometry{margin=2.5cm}

\title{\textbf{Simulación Proporcional de Reparto de Torta\\
Método del Último Reductor}}
\author{Trabajo Práctico – Moroni Giancarlo, Pañale Agustín}
\date{\today}

\begin{document}
\maketitle

\section{Objetivo}
Evaluar mediante simulación la garantía proporcional
(\(\forall p:\ U_p \ge \tfrac13\)) en un reparto de una torta
unidimensional de \(T\) pedazos entre tres jugadores,
utilizando el protocolo \emph{Last–Diminisher} con cuota fija \(1/3\).

\section{Descripción del algoritmo}
Se implementó en Python la clase \texttt{CakeDivisionSimulator}
con los siguientes métodos clave:
\begin{itemize}
  \item \textbf{generate\_cake(T)}: arreglo aleatorio de long.\(T\)
        con componentes \texttt{'A'} o \texttt{'B'}.
  \item \textbf{generate\_random\_preferences()}: valores
        \(v_{p,A},v_{p,B}\sim U(0.1,2.0)\) para cada jugador \(p\).
  \item \textbf{normalize\_utilities(cake,prefs)}: escala de modo
        que la suma de valores del cake entero sea \(1\).
  \item \textbf{last\_diminisher\_algorithm(cake,prefs)}:
        \begin{enumerate}
          \item Cada ronda sigue: el primer jugador propone un
                corte donde su valor acumulado alcance \(\tfrac13\);
          \item los demás pueden “reducir” si valoran ese trozo
                por encima de \(\tfrac13\);
          \item el último que reduce se lleva el trozo;
          \item se repite con el resto hasta asignar a todos.
        \end{enumerate}
  \item \textbf{run\_simulation(T,N)}: repite \(N\) veces
        el reparto para tortas de tamaño \(T\), guarda utilidades
        y desviaciones medias \(\frac{1}{3}\sum|U_p-\tfrac13|\).
\end{itemize}

\section{Resultados}
Se ejecutó la simulación con \(T=300\), \(N=50\).  
La Figura~\ref{fig:resultados} muestra:
\begin{itemize}
  \item \emph{Utilities per Iteration}: cada jugador oscila
        muy cerca de \(1/3\), sin caer por debajo.
  \item \emph{Utility Distribution}: histograma concentrado
        a la derecha de la línea \(1/3\).
  \item \emph{Deviation per Iteration}: desviaciones típicas
        en torno a \(0.003\)–\(0.006\).
  \item \emph{Boxplot Utilities}: cajas centradas en \(\ge0.333\)
        con colas superiores moderadas.
\end{itemize}

\begin{figure}[ht]
  \centering
  \includegraphics[width=\textwidth]{cakecutting.png}
  \caption{Simulación con método Last–Diminisher fijo (\(T=300,N=50\)).}
  \label{fig:resultados}
\end{figure}

\section{Conclusiones}
\begin{itemize}
  \item El protocolo garantiza proporcionalidad: ningún jugador
        recibe menos de \(\tfrac13\) salvo error discretizado \(1/T\).
  \item Las utilidades mínimas observadas rondan \(0.332\), dentro
        del margen \(\pm1/300\approx0.0033\).
  \item La variabilidad residual se debe a la granularidad discreta.
  \item Para reducir aún más la dispersión, aumentar \(T\) o usar
        algoritmos de equidad más fuertes (p.ej.\ Selfridge–Conway).
\end{itemize}

\end{document}
